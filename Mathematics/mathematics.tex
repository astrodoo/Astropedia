\chapter{Mathematical physics}

\section{Mathematical physics}

\subsection{Moment of Inertia}

Moment of inertia is the name given to rotational inertia, the rotational analog of mass for
linear motion.  It appears in the relationships for the dynamics of rotational motion. The
moment of inertia must be specified with respect to a chosen axis of rotation. For a point
mass, the moment of inertia is just the mass times the square of perpendicular distance to
the rotation axis, $I=m\,r^{2}$. That point mass relationship becomes the basis for all other
moment of inertia since any object can be built up from a collection of point masses. For the
extended source, we have to integrate through the volume,

\begin{equation}
   I = \int dI = \int^{M}_{0}r^{2}dm
\end{equation}

\subsubsection{sphere}

To derive the moment of inertia of a solid sphere of uniform density and radius R,
\begin{equation}
   dI = r^{2}_{\perp} dm = r^{2}_{\perp}\rho\,dV
\end{equation}
where $r_{\perp}$ is the perpendicular distance to a point at $\vec{r}$ from the axis of rotation.
Therefore, $r_{\perp}= |\vec{r}|\,\sin{\theta}$. Integrating over the volume,
\begin{eqnarray}
   I &=& \int \int \int_{V} \rho\,r^{2}_{\perp}\,dV = \rho \int^{2\pi}_{0}\int^{\pi}_{0}\int^{R}_{0} \left( r\,\sin{\theta} \right)^{2}\,r^{2}\,\sin{\theta}drd\theta d\phi \nonumnext
     &=& 2\pi\rho \int^{\pi}_{0} \sin^{3}{\theta}d\theta \int^{R}_{0}r^{4}dr ~~~~~~~~~ \leftarrow \textrm{substituting } u=\cos{\theta} \nonumnext
     &=& 2\pi\rho \frac{R^{5}}{5}\int^{1}_{-1} \left( 1 - u^{2} \right) du = \frac{4}{3}\pi R^{3}\rho\left( \frac{2}{5} R^{2}\right) \nonumnext
     &=& \frac{2}{5}M\,R^{2}
\end{eqnarray}
where $M$ is the total mass of the sphere.


\bigskip
\subsection{Potential of Uniform Sphere}\label{subsec:potunisp}
To be updated
\begin{equation}
    U = -\frac{3}{5} \frac{G\,M}{R}
\end{equation}

\bigskip
\subsection{Fourier decomposition}


