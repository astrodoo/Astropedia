\chapter{Gravity}

\section{Gravity}
\subsection{Newtonian gravity}
A mass m' at position $\rb'$ exerts on any other mass m at position $\rb$ an attractive
force $\Fb=m {\bf g}(\rb)$; the gravitational acceleration {\bf g(r)} can be written as the 
gradient of a potential function, ${\bf g} = -\nabla \Psi$, where
\begin{equation}\label{eq:grav}
   \Psi = \frac{G m'}{|\rb - \rb'|},
\end{equation}
\begin{equation}\label{eq:gravf}
   \therefore |\nabla \Psi| = \frac{G m'}{|\rb - \rb'|^2}.
\end{equation}
Let S be a spherical surface of radius ${\rb - \rb'}$ centered at {\bf r'}, then we have
\begin{equation}
   \int_{S} \nb \cdot \nabla \Psi dS = 4 \pi G m'.
\end{equation}
It can also be verified directly that the gravitational potential of our point mass (eq 
\ref{eq:grav}) satisfies $\nabla \cdot \nabla \Psi \equiv \nabla^2 \Psi = 0$ (the Laplace equation)
everywhere except for just one point, $\rb=\rb'$. By using divergence theorem, eq \ref{eq:gravf} can 
be expressed as,
\begin{equation}\label{eq:poisson}
   \int_{V} \nabla^2 \Psi\, dV = 4 \pi G \int_{V} \rho \,dV,
\end{equation}
where V is volume inside S. Since V is arbitrary, this equation can be rewritten as a 
partial differential equation, {\bf Poisson's equation:}
\begin{equation}
   \nabla^2 \Psi = 4 \pi G \rho
\end{equation}

\bigskip
\subsection{Simple models of astrophysical fluid and their motions}
In the previous lecture we established the continuity equation (\ref{eq:continuity2}), the momentum
equation (\ref{eq:momentum}), the energy equation (\ref{eq:totE}), and Poisson's equation (\ref{eq:poisson}).
Assuming that the only body forces are due to self-gravity, so that $\fb=-\nabla \Psi$ in 
eq. \ref{eq:momentum}, these equations are:
\begin{empheq}[left=\empheqlbrack,right=\empheqrbrack]{align}
    \frac{D\rho}{Dt}+\rho\nabla\cdot\ub &= 0, \\
    \rho \frac{D \ub}{Dt} &= -\nabla P - \rho \nabla \Psi, \label{eq:mom2}\\
    \frac{DU}{Dt}-\frac{P}{\rho^2}\frac{D\rho}{Dt} &= \varepsilon - \frac{1}{\rho} \nabla \cdot \Fb,
      \label{eq:totE2} \\
    \nabla^2 \Psi &= 4 \pi G \rho. \label{eq:poisson2}
\end{empheq}
Note that these contain seven dependent variables ($\rho, P, \Psi, U$,and three components
of $\ub$).  The three equations from momentum equation, together with the rest equations,
provide six equations, and a seventh is the equation of state (\eg that for an ideal gas)
which provides a relation between any three thermodynamic state variables, so that for instance
the internal energy U and temperature T can be written in terms of P and $\rho$.($\varepsilon$
and $\Fb$ are assumed to be known functions of the other variables). Thus one might hope in
principle to solve these equations, given suitable boundary conditions. In practice this set of
equations is intractable to exact solution, and one must resort to numerical solutions. Even
these can be extremely problematic so that, for example, understanding turbulent flows is
still a very challenging research area. Moreover, an analytic solution to a somewhat idealized
problem may teach one much more than a numerical solution. One useful idealization is where we
assume that the fluid velocity and all time derivatives are zero. These are called equilibrium
solutions and describe a steady an astrophysical system evolves may be very long, so that at
any particular time the state of many astrophysical fluid bodies may be well represented by an
equilibrium model. Even when the dynamical behaviour of the body is important, it can often be
described in terms of small departures from an equilibrium state.

\bigskip
\subsection{Hydrostatic equilibrium for a self-gravitating body}

If we suppose that $\ub=0$ every where, and that all quantities are independent of time,
eq. \ref{eq:mom2} becomes
\begin{equation}\label{eq:hse}
  \nabla P + \rho \nabla \Psi = 0
\end{equation}
the continuity equation becomes trivial. A fluid satisfying eq. \ref{eq:hse} is said to
be in {\bf hydrostatic equilibrium}. If it is self-gravitating (so that $\Psi$ is determined
by the density distribution within the fluid), then eq. \ref{eq:poisson2} must also be satisfied.

Putting $\ub=0$ and $\partial/\partial t$=0, \ie $D/Dt=0$, in eq. \ref{eq:totE2}, we obtain
that the heat sources given by $\varepsilon$ must be exactly balance by the heat flux term
$\rho^{-1}\nabla \cdot \Fb$. If this holds, then the fluid is also said to be in thermal
equilibrium (See \S \ref{sec:thermaleq}).

\bigskip
\subsection{The formation of protostars}

\subsubsection{Stellar time scales}

\textbf{a) dynamical time scale}

The length of time over which changes in one part of a body can be communicated to the rest of that body.
That is also called, freefall time scale.

Assuming $|dP/dr| \ll G M_{r}\rho/r^{2}$, where $M_{r}$ is the mass of the spherical cloud,
\begin{equation}
   \frac{d^2 r}{dt^{2}} = -G \frac{M_{r}}{r^{2}} = -\frac{G}{r^{2}} \frac{4\pi}{3} \rho_{0} r_{0}^{3}, 
\end{equation}
where, $r_{0}$ and $\rho_{0}$ is the initial radius and density of the sphere. Multiplying the velocity 
of the surface  of the sphere for both sides,
\begin{equation}
   \frac{dr}{dt}\frac{d^2 r}{dt^{2}} = -\frac{G}{r^{2}} \frac{4\pi}{3} \rho_{0} r_{0}^{3} \frac{dr}{dt},
\end{equation}
which can be integrated to give
\begin{equation}
   \frac{1}{2}\left( \frac{dr}{dt} \right)^{2} = \left( \frac{4\pi}{3}G\rho_{0}r_{0}^{3} \right)\frac{1}{r} + C_{1}.
\end{equation}
where $C_{1}$ can be evaluated by, $dr/dt=0$ when $r=r_{0}$. This gives
\begin{equation}
   C_{1} = -\frac{4\pi}{3} G \rho_{0} r_{0}^2.
\end{equation}
therefore,
\begin{equation}
   \frac{dr}{dt} = - \left[ \frac{8\pi}{3}G \rho_{0} r_{0}^{2} \left( \frac{r_{0}}{r} -1 \right) \right]^{1/2}.
\end{equation}
Substituting $\theta \equiv r/r_{0}$ and $K \equiv \left( \frac{8\pi}{3} G \rho_{0} \right)^{1/2}$ gives,
\begin{equation}
   \frac{d\theta}{dt} = - K \left( \frac{1}{\theta} -1 \right)^{1/2}.
\end{equation}
Making another substitution, $\theta \equiv \cos^2{\phi}$, then
\begin{equation}
   \cos^{2}{\phi} \frac{d\phi}{dt} = \frac{K}{2}.
\end{equation}
This will be integrated to yield
\begin{equation}
   \frac{\sin{2\phi}}{4} + \frac{\phi}{2} = \frac{K}{2}t + C_{2}.
\end{equation}
where $C_{2}$ can be evaluated by, $r=r_{0}$ when $t=0$ implying $\theta=1$ or $\phi=0$ at the beginning of the collapse,
then gives $C_{2}=0$.

Consequently, the freefall time scale or dynamical time scale can be calculated by, $\theta=0$ or $\phi = \pi/2$,
\begin{eqnarray}\label{eq:fftime}
   t_{dyn} = t_{ff} &=& \frac{\pi}{2\,K} \nonumnext
                    &=& \left( \frac{3\pi}{32} \frac{1}{G \rho_{0}}\right)^{1/2}.
\end{eqnarray}

\textbf{b) thermal time scale}

The time scale on which the star would contract if its nuclear energy sources were turned off. And 
it is also called, kelvin-Helmholtz time scale:

\begin{equation}
   t_{KH} \approx \frac{G M^{2}/R_{*}}{L}.
\end{equation}

\textbf{c) nuclear time scale}
The heat released by fusing a mass $\triangle M c^{2}$. Therefore the time required to exhaust all the star's 
hydrogen is
\begin{equation}
   t_{nuc} = \frac{0.007 M c^{2}}{L}
\end{equation}

\subsubsection{Jean's Instability}\label{subsubsec:Jean}

Two methods will be described to derive Jean's Mass.

\textbf{a) From virial theorem}

The potential energy is,
\begin{equation}
   dU = G \frac{M_{interior}M_{shell}}{r}.
\end{equation}
Integrating the equation, 
\begin{equation}
   U = G \int^{0}_{R} \frac{4/3\pi r^{3}\rho ~4\pi r^{2} \rho\,dr}{r} = -\frac{3}{5} \frac{G M^{2}}{R},
\end{equation}
where $\rho = M / (\frac{4}{3}\pi R^{3})$.

The kinetic energy is
\begin{equation}
   K = N_{H}~\frac{3}{2}kT = \frac{M}{\mu m_{H}}~\frac{3}{2}kT.
\end{equation}
Using a virial theorem, 2K+U=0,
\begin{equation}
   \frac{3}{5}\frac{G M^{2}}{R} = 3 \frac{M k T}{\mu m_{H}},
\end{equation}
therefore, the Jean's mass can be calculated to
\begin{equation}
   M_{J} = \left( \frac{5 k T}{G\mu m_{H}} \right)^{3/2} \left( \frac{3}{4\pi \rho} \right)^{1/2},
\end{equation}
because $R=\left( M/(\frac{4}{3}\pi \rho) \right)^{1/3}$. The following length is
\begin{equation}
   R_{J} = \left( \frac{M_{J}}{\frac{4}{3}\pi\rho} \right)^{1/3} = \left( \frac{5 k T}{G\mu m_{H}} \right)^{1/2} \left( \frac{3}{4\pi\rho} \right)^{1/2},
\end{equation}
where the Jean's length, $\lambda_{J} = 2 R_{J}$.

\textbf{b) From freefall time}

The freefall time scale is, from Eqn. \ref{eq:fftime}, 
\begin{equation}
  t_{ff} = \left( \frac{3\pi}{32} \frac{1}{G \rho_{0}}\right)^{1/2}.
\end{equation}
In order to collapse the cloud, the freefall time scale should be less than the crossing time scale which 
is the time scale in which the information propagates from edge to edge, that is
\begin{equation}
   t_{cs} = \frac{\lambda_{J}}{Cs},
\end{equation}
where $Cs$ is the sound speed of $\sqrt{\frac{kT}{\mu m_{H}}}$. As a result,
\begin{equation}
   \left( \frac{3\pi}{32} \frac{1}{G \rho_{0}}\right)^{1/2} = \frac{\lambda_{J}}{Cs}
\end{equation}
or,
\begin{equation}
   \lambda_{J} = \left( \frac{3\pi k T}{32 G \rho \mu m_{H}} \right)^{1/2}.
\end{equation}

\bigskip
\subsection{Special relativity}
to be described

\bigskip
\subsection{General relativity}
to be described
section{Gravity}
\subsection{Newtonian gravity}
A mass m' at position $\rb'$ exerts on any other mass m at position {\bf r} an attractive
force $\Fb=m{ {\bf g}(\rb)}$; the gravitational acceleration ${\bf g}(\rb)$ can be written as the 
gradient of a potential function, ${\bf g} = -\nabla \Psi$, where
\begin{equation}\label{eq:grav}
   \Psi = \frac{G m'}{|\rb - \rb'|},
\end{equation}
\begin{equation}\label{eq:gravf}
   \therefore |\nabla \Psi| = \frac{G m'}{|\rb - \rb'|^2}.
\end{equation}
Let S be a spherical surface of radius $|\rb - \rb'|$ centered at $\rb'$, then we have
\begin{equation}
   \int_{S} \nb \cdot \nabla \Psi dS = 4 \pi G m'.
\end{equation}
It can also be verified directly that the gravitational potential of our point mass (eq 
\ref{eq:grav}) satisfies $\nabla \cdot \nabla \Psi \equiv \nabla^2 \Psi = 0$ (the Laplace equation)
everywhere except for just one point, $\rb=\rb'$. By using divergence theorem, eq \ref{eq:gravf} can 
be expressed as,
\begin{equation}\label{eq:poisson}
   \int_{V} \nabla^2 \Psi\, dV = 4 \pi G \int_{V} \rho \,dV,
\end{equation}
where V is volume inside S. Since V is arbitrary, this equation can be rewritten as a 
partial differential equation, {\bf Poisson's equation:}
\begin{equation}
   \nabla^2 \Psi = 4 \pi G \rho
\end{equation}

\bigskip
\subsection{Simple models of astrophysical fluid and their motions}
In the previous lecture we established the continuity equation (\ref{eq:continuity2}), the momentum
equation (\ref{eq:momentum}), the energy equation (\ref{eq:totE}), and Poisson's equation (\ref{eq:poisson}).
Assuming that the only body forces are due to self-gravity, so that $\fb=-\nabla \Psi$ in 
eq. \ref{eq:momentum}, these equations are:
\begin{empheq}[left=\empheqlbrack,right=\empheqrbrack]{align}
    \frac{D\rho}{Dt}+\rho\nabla\cdot\ub &= 0, \\
    \rho \frac{D \ub}{Dt} &= -\nabla P - \rho \nabla \Psi, \label{eq:mom2}\\
    \frac{DU}{Dt}-\frac{P}{\rho^2}\frac{D\rho}{Dt} &= \varepsilon - \frac{1}{\rho} \nabla \cdot \Fb,
      \label{eq:totE2} \\
    \nabla^2 \Psi &= 4 \pi G \rho. \label{eq:poisson2}
\end{empheq}
Note that these contain seven dependent variables ($\rho, P, \Psi, U$,and three components
of $\ub$).  The three equations from momentum equation, together with the rest equations,
provide six equations, and a seventh is the equation of state (\eg that for an ideal gas)
which provides a relation between any three thermodynamic state variables, so that for instance
the internal energy U and temperature T can be written in terms of P and $\rho$.($\varepsilon$
and $\Fb$ are assumed to be known functions of the other variables). Thus one might hope in
principle to solve these equations, given suitable boundary conditions. In practice this set of
equations is intractable to exact solution, and one must resort to numerical solutions. Even
these can be extremely problematic so that, for example, understanding turbulent flows is
still a very challenging research area. Moreover, an analytic solution to a somewhat idealized
problem may teach one much more than a numerical solution. One useful idealization is where we
assume that the fluid velocity and all time derivatives are zero. These are called equilibrium
solutions and describe a steady an astrophysical system evolves may be very long, so that at
any particular time the state of many astrophysical fluid bodies may be well represented by an
equilibrium model. Even when the dynamical behaviour of the body is important, it can often be
described in terms of small departures from an equilibrium state.

\bigskip
\subsection{Hydrostatic equilibrium for a self-gravitating body}

If we suppose that $\ub=0$ every where, and that all quantities are independent of time,
eq. \ref{eq:mom2} becomes
\begin{equation}\label{eq:hse}
  \nabla P + \rho \nabla \Psi = 0
\end{equation}
the continuity equation becomes trivial. A fluid satisfying eq. \ref{eq:hse} is said to
be in {\bf hydrostatic equilibrium}. If it is self-gravitating (so that $\Psi$ is determined
by the density distribution within the fluid), then eq. \ref{eq:poisson2} must also be satisfied.

Putting $\ub=0$ and $\partial/\partial t$=0, \ie D/Dt=0, in eq. \ref{eq:totE2}, we obtain
that the heat sources given by $\varepsilon$ must be exactly balance by the heat flux term
$\rho^{-1}\nabla \cdot \Fb$. If this holds, then the fluid is also said to be in thermal
equilibrium (See \S \ref{sec:thermaleq}).

\bigskip
\subsection{The formation of protostars}

\subsubsection{Stellar time scales}

\textbf{a) dynamical time scale}

The length of time over which changes in one part of a body can be communicated to the rest of that body.
That is also called, freefall time scale.

Assuming $|dP/dr| \ll G M_{r}\rho/r^{2}$, where $M_{r}$ is the mass of the spherical cloud,
\begin{equation}
   \frac{d^2 r}{dt^{2}} = -G \frac{M_{r}}{r^{2}} = -\frac{G}{r^{2}} \frac{4\pi}{3} \rho_{0} r_{0}^{3}, 
\end{equation}
where, $r_{0}$ and $\rho_{0}$ is the initial radius and density of the sphere. Multiplying the velocity 
of the surface  of the sphere for both sides,
\begin{equation}
   \frac{dr}{dt}\frac{d^2 r}{dt^{2}} = -\frac{G}{r^{2}} \frac{4\pi}{3} \rho_{0} r_{0}^{3} \frac{dr}{dt},
\end{equation}
which can be integrated to give
\begin{equation}
   \frac{1}{2}\left( \frac{dr}{dt} \right)^{2} = \left( \frac{4\pi}{3}G\rho_{0}r_{0}^{3} \right)\frac{1}{r} + C_{1}.
\end{equation}
where $C_{1}$ can be evaluated by, $dr/dt=0$ when $r=r_{0}$. This gives
\begin{equation}
   C_{1} = -\frac{4\pi}{3} G \rho_{0} r_{0}^2.
\end{equation}
therefore,
\begin{equation}
   \frac{dr}{dt} = - \left[ \frac{8\pi}{3}G \rho_{0} r_{0}^{2} \left( \frac{r_{0}}{r} -1 \right) \right]^{1/2}.
\end{equation}
Substituting $\theta \equiv r/r_{0}$ and $K \equiv \left( \frac{8\pi}{3} G \rho_{0} \right)^{1/2}$ gives,
\begin{equation}
   \frac{d\theta}{dt} = - K \left( \frac{1}{\theta} -1 \right)^{1/2}.
\end{equation}
Making another substitution, $\theta \equiv \cos^2{\phi}$, then
\begin{equation}
   \cos^{2}{\phi} \frac{d\phi}{dt} = \frac{K}{2}.
\end{equation}
This will be integrated to yield
\begin{equation}
   \frac{\sin{2\phi}}{4} + \frac{\phi}{2} = \frac{K}{2}t + C_{2}.
\end{equation}
where $C_{2}$ can be evaluated by, $r=r_{0}$ when $t=0$ implying $\theta=1$ or $\phi=0$ at the beginning of the collapse,
then gives $C_{2}=0$.

Consequently, the freefall time scale or dynamical time scale can be calculated by, $\theta=0$ or $\phi = \pi/2$,
\begin{eqnarray}\label{eq:fftime}
   t_{dyn} = t_{ff} &=& \frac{\pi}{2\,K} \nonumnext
                    &=& \left( \frac{3\pi}{32} \frac{1}{G \rho_{0}}\right)^{1/2}.
\end{eqnarray}

\textbf{b) thermal time scale}

The time scale on which the star would contract if its nuclear energy sources were turned off. And 
it is also called, kelvin-Helmholtz time scale:

\begin{equation}
   t_{KH} \approx \frac{G M^{2}/R_{*}}{L}.
\end{equation}

\textbf{c) nuclear time scale}
The heat released by fusing a mass $\triangle M c^{2}$. Therefore the time required to exhaust all the star's 
hydrogen is
\begin{equation}
   t_{nuc} = \frac{0.007 M c^{2}}{L}
\end{equation}

\subsubsection{Jean's Instability}\label{subsubsec:Jean}

Two methods will be described to derive Jean's Mass.

\textbf{a) From virial theorem}

The potential energy is,
\begin{equation}
   dU = G \frac{M_{interior}M_{shell}}{r}.
\end{equation}
Integrating the equation, 
\begin{equation}
   U = G \int^{0}_{R} \frac{4/3\pi r^{3}\rho ~4\pi r^{2} \rho\,dr}{r} = -\frac{3}{5} \frac{G M^{2}}{R},
\end{equation}
where $\rho = M / (\frac{4}{3}\pi R^{3})$.

The kinetic energy is
\begin{equation}
   K = N_{H}~\frac{3}{2}kT = \frac{M}{\mu m_{H}}~\frac{3}{2}kT.
\end{equation}
Using a virial theorem, 2K+U=0,
\begin{equation}
   \frac{3}{5}\frac{G M^{2}}{R} = 3 \frac{M k T}{\mu m_{H}},
\end{equation}
therefore, the Jean's mass can be calculated to
\begin{equation}
   M_{J} = \left( \frac{5 k T}{G\mu m_{H}} \right)^{3/2} \left( \frac{3}{4\pi \rho} \right)^{1/2},
\end{equation}
because $R=\left( M/(\frac{4}{3}\pi \rho) \right)^{1/3}$. The following length is
\begin{equation}
   R_{J} = \left( \frac{M_{J}}{\frac{4}{3}\pi\rho} \right)^{1/3} = \left( \frac{5 k T}{G\mu m_{H}} \right)^{1/2} \left( \frac{3}{4\pi\rho} \right)^{1/2},
\end{equation}
where the Jean's length, $\lambda_{J} = 2 R_{J}$.

\textbf{b) From freefall time}

The freefall time scale is, from Eqn. \ref{eq:fftime}, 
\begin{equation}
  t_{ff} = \left( \frac{3\pi}{32} \frac{1}{G \rho_{0}}\right)^{1/2}.
\end{equation}
In order to collapse the cloud, the freefall time scale should be less than the crossing time scale which 
is the time scale in which the information propagates from edge to edge, that is
\begin{equation}
   t_{cs} = \frac{\lambda_{J}}{Cs},
\end{equation}
where $Cs$ is the sound speed of $\sqrt{\frac{kT}{\mu m_{H}}}$. As a result,
\begin{equation}
   \left( \frac{3\pi}{32} \frac{1}{G \rho_{0}}\right)^{1/2} = \frac{\lambda_{J}}{Cs}
\end{equation}
or,
\begin{equation}
   \lambda_{J} = \left( \frac{3\pi k T}{32 G \rho \mu m_{H}} \right)^{1/2}.
\end{equation}

\bigskip
\subsection{Special relativity}
to be described

\bigskip
\subsection{General relativity}
to be described

%\bibliographystyle{apj}
%\bibliography{citations}
