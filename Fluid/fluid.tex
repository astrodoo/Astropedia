\chapter{Fluid Dynamics}

\section{Fluid Dynamics}

\begin{itemize}
   \item $\partial / \partial t$: the rate of change of some physical quantity with respect to time at a 
fixed position in space.

   \item $D /Dt$ (the material derivative): the rate of change of some quantity with respect to time but
traveling along with the fluid. 

Let $f$ be any quantity (e.g. density), then
\begin{equation}\label{eq:lageu}
   \frac{Df}{Dt} = \frac{\partial f}{\partial t} + u \cdot \nabla f,
\end{equation}
where $u(r,t)$ is the velocity of the fluid at position $r$ and time $t$.

\bigskip
\subsection{Conservations}

\subsubsection{The continuity equation}
Consider a volume $V$, which is fixed in space. The total mass of fluid in V is $\int_{V} \rho d V$.
The time derivative of the mass in $V$ is the mass flux into V across its surface S, i.e.
\begin{equation}
   \frac{d}{dt}\int_{V} \rho dV = - \int_{S} (\rho \ub)\cdot \nb \, dS,
\end{equation}
where $\nb$ is the outward normal to the surface S. By using the divergence theorem, we obtain
\end{itemize}

\begin{equation}
   \int_{V} \frac{\partial \rho}{\partial t} d V = - \int_{S} \rho \ub \cdot \nb dS
   =-\int_{V} \nabla \cdot (\rho \ub) dV.
\end{equation}
\begin{equation}\label{eq:continuity1}
  \therefore \, \frac{\partial \rho}{\partial t} + \nabla \cdot (\rho \ub) = 0.
\end{equation}
This is the continuity(or mass conservation) equation. Using eq \ref{eq:lageu} this can also be 
written as 
\begin{equation}\label{eq:continuity2}
   \frac{D \rho}{D t} + \rho \nabla \cdot \ub=0.
\end{equation}

\bigskip
\subsubsection{The momentum equation}
By analogy, one can derive a momentum equation, or equation of motion, for the fluid by considering
the rate of change of the total momentum of the fluid inside a volume V. The momentum of the fluid
in V is $\int_{V} \rho \ub \,dV$, and the rate of change of this momentum is equal to the net
force acting on the fluid in volume V. These net force consists of two kinds; one is body force, such
as gravity, which act on the particles inside V, and the other is surface force - forces exerted on 
the surface S of V by the surrounding fluid. 

The former body force can be expressed as
\begin{equation}
  \int_{V} \rho \fb \, dV,
\end{equation}
where $\fb$ is the body force per unit mass (\ie dimension: acceleration). 
The latter surface force is
\begin{equation}
  -\int_{S} P \nb\,dS,
\end{equation}
where P is the pressure. Equating force to change of momentum we obtain
\begin{equation}
  \frac{d}{dt} \int_{V} \rho \ub \,dV = -\int_{S} P \nb \,dS
                                        +\int_{V} \rho \fb \, dV.
\end{equation}
Since $\rho \, dV$, the mass of a fluid element, is invariant following the motion,
\begin{equation}
  \frac{d}{dt} \int_{V} \rho \ub \,dV = \int_{V} \rho \frac{D\ub}{Dt} \,dV 
\end{equation}
and hence, applying the divergence theorem to the surface integral, we obtain
\begin{equation}
  \int_{V} \rho \frac{D\ub}{Dt} \,dV = \int_{V} \left( -\nabla P + \rho \fb \right)\, dV.
\end{equation}
\begin{equation}\label{eq:momentum}
  \therefore \, \rho \frac{D\ub}{Dt} = \rho \left( \frac{\partial \ub}{\partial t} 
              + \left( \ub \cdot \nabla \right) \ub \right) = -\nabla P + \rho \fb.
\end{equation}
This is the momentum equation for an inviscid fluid. Taking into account the viscous forces
would add the right-hand side of the momentum equation with an additional term 
$\mu \left( \nabla^{2} \ub + \frac{1}{3}\nabla (\nabla \cdot \ub)\right)$, where
$\mu$ is dynamic viscosity.

\bigskip
\subsubsection{The Energy equation}
Taking a dot product of the equation of motion for a fluid, eq. \ref{eq:momentum}, with the fluid
velocity $\ub$ yields

\begin{equation}\label{eq:kinE}
   \frac{D}{Dt}\left( \frac{1}{2} \ub^2 \right) = -\frac{1}{\rho} \ub \cdot \nabla P
      + \ub \cdot \fb.
\end{equation}
Eq. \ref{eq:kinE} says that the rate of change of the kinetic energy of a unit mass of fluid
is equal to the rate at which work is done on the fluid by pressure and body forces. This is 
sometimes called the {\bf mechanical energy equation}.

An equation for the total energy - kinetic and internal thermal energy - can be derived in the same
manner as was the momentum equation. Let the internal energy per unit mass of fluid be U. Then the 
rate of change of kinetic plus internal energy of a material volume (\ie one moving with the fluid) 
must be equal to the rate of work done on the fluid by surface and body forces, plus the rate at
which heat is added to the fluid. Heat can be added in two ways: one is by its being generated at
a rate $\varepsilon$ per unit mass within the fluid volume (\eg by nuclear reactions), while the 
second is by the flux of heat {\bf F} into the volume from the surroundings (\eg by radiation). Thus
\begin{equation}
   \frac{d}{dt}\int_{V}\left( \frac{1}{2} \ub^2 + U \right) \rho \, dV
  = \int_{S}\ub \cdot (-P \nb)\, dS + \int_{V} \ub \cdot \fb \rho\,dV
   +\int_{V}\varepsilon\rho\,dV - \int_{S} \Fb \cdot \nb \,dS.
\end{equation}

In the same way as for the momentum equation, one rewrites all the surface integrals in this  equation
as volume integrals, using the divergence theorem. The resulting equation holds for an arbitrary 
volume V and so one deduces that
\begin{equation}\label{eq:totE}
  \rho \left( \frac{D}{Dt} \left( \frac{1}{2} \ub^2 + \frac{DU}{Dt} \right) \right)
   = -\nabla \cdot (P \ub) + \rho \ub \cdot \fb + \rho \varepsilon - \nabla \cdot \Fb.
\end{equation}
One can derive an equation for the thermal energy alone by dividing eq. \ref{eq:totE} by the 
density and then subtracting the kinetic energy equation, eq. \ref{eq:kinE}:
\begin{equation}\label{eq:thermalE}
  \frac{DU}{Dt} = \frac{P}{\rho^2}\frac{D\rho}{Dt}+\varepsilon-\frac{1}{\rho}\nabla\cdot \Fb.
\end{equation}
Note that the divergence of $\nabla \cdot \ub$ has been replaced by $-\rho^{-1} D\rho/Dt$ using the 
continuity equation, eq. \ref{eq:continuity2}.

\bigskip
\subsection{Shock}
\subsubsection{Basic properties}

\noi{\bf ideal gas}

The pressure is
\begin{equation}
    P = n\,k\,T = \frac{\rho\,k\,T}{\mu\,m_{H}}
\end{equation}
where $\mu$ is mean molecular weight.

\medskip
\noi{\bf Mean molecular weight}
The mean particle mass is
\begin{equation}
    \bar{m} = \frac{n_{1}m_{1} + n_{2}m_{2} + n_{3}m_{3}+...}{n_{1}+n_{2}+n_{3}+...} = \frac{\rho}{n} = \mu\,m_{H},
\end{equation}
where $\mu$ is mean molecular weight.

The abundance of medium can be expressed as $(X,Y,Z)$, which represent hydrogen, helium and heavy
elements, respectively.  By definition, the summation of them should be $\displaystyle\sum_{i}
X_{i} = X+Y+Z = 1$. The number density of hydrogen, helium, or an element of atomic mass number $A$ will be
\begin{equation}
    n_{H} = \frac{X\,\rho}{m_{H}},~~n_{He}=\frac{Y\,\rho}{4\,m_{H}},~~n_{A}=\frac{Z_{A}\,\rho}{A\,m_{H}}
\end{equation}

In neutral limit, heavy elements is negligible. The number density of gas will be
\begin{equation}
    n = n_{H} + n_{He} + n_{A} \simeq n_{H} + n_{He} = \frac{\rho}{m_{H}} \left( X+\frac{Y}{4} \right),
\end{equation}
in other word, the mean molecular weight will be
\begin{equation}
    \frac{1}{\mu} = X+\frac{Y}{4}.
\end{equation}


In full-ionization limit, hydrgen produces 2 atoms (1 ion + 1 electron) and helium produces 3 atoms (1 ion + 2 electrons) 
and the heavy enough atoms produce the atomic number (ions+electrons) close to $A/2$.
\begin{equation}
    n \simeq n_{H} + n_{He} + \sum \frac{A}{2}\,n_{A} = \frac{\rho}{m_{H}} \left( 2X+\frac{3Y}{4}+\frac{1}{2}Z \right),
\end{equation}
in other word, the mean molecular weight will be
\begin{equation}
    \frac{1}{\mu} = 2X+\frac{3Y}{4}+\frac{1}{2}Z = \frac{1}{2}\left( 3X + \frac{Y}{2} + 1 \right).
\end{equation}

For instance, if we apply the solar abundance (X=0.7, Y=0.28, Z=0.02) to the formula in two limits above,
the mean molecular weight will be $\mu_{\rm solar,non-ionized}=1.3$ \& $\mu_{\rm solar,ionized} =0.62$.

The general expression of mean molecular weight is
\begin{equation}
    \frac{1}{\mu} = \frac{1}{\mu_{I}} + \frac{1}{\mu_{e}} = \sum\frac{X_{i}}{A_{i}} + \sum\frac{f_{i}Z_{i}X_{i}}{A_{i}},
\end{equation}
where $\mu_{I},\,\mu_{e}$ is the mean molecular weight of ion and electron, respectively, $f_{i}$ is an ionization fraction.


\medskip
\noi{\bf Sound Speed $C_{s}$}

In adiabatic situation $P = A\, \rho^{\gamma}$, where $A$ is constant, the sound speed is
\begin{equation}
    C_{s}^{2} = \frac{\partial P}{\partial \rho} = \gamma \frac{P}{\rho}.
\end{equation}

In isothermal situation ($\gamma=1$),
\begin{equation}
    C_{s}^{2} = \frac{P}{\rho} = \frac{k\,T}{\mu\,m_{H}}.
\end{equation}


\subsubsection{Rankin-Hugoniot Condition}
According to the conservation equations for mass, momentum and enery, the flow across
the shock front has to satisfy the following jump conditions:

\begin{eqnarray}
    \rho_{2}\,u_{2} &=& \rho_{1}\,u_{1}    \\
    \rho_{2}\,u_{2}^{2} + P_{2} &=& \rho_{1}\,u_{1}^{2} + P_{1}    \\
    \frac{1}{2}u_{2}^{2} + h_{2} &=& \frac{1}{2}u_{1}^{2} + h_{1}
\end{eqnarray}
where the subscription 1 and 2 represent upstream and downstream, respectively, and 
$h$ denotes the specific enthalpy which for a perfect gas satisfies the constituve relations:
\begin{equation}
    h=\frac{\gamma}{\gamma-1}\frac{P}{\rho} = \frac{\gamma}{\gamma-1}\frac{k\,T}{\mu\,m_{H}}.
\end{equation}
The upstream Mach number $\mach_{1} \equiv u_{1}/C_{s}$, where the sound speed of 
$C_{s} = \sqrt{\gamma\,P/\rho}$, the ratio of physical properties between downstream and upstream will be:
\begin{eqnarray}
    \frac{\rho_{2}}{\rho_{1}} &=& \frac{u_{1}}{u_{2}} = \frac{(\gamma+1)\mach_{1}^{2}}{(\gamma+1) + (\gamma-1)(\mach_{1}^{2}-1)} \\
    \frac{P_{2}}{P_{1}} &=& \frac{(\gamma+1) + 2\gamma(\mach_{1}^{2}-1)}{\gamma+1} \\
    \frac{T_{2}}{T_{1}} &=& \frac{[ (\gamma+1) + 2\gamma(\mach_{1}^{2}-1)] [(\gamma+1) + (\gamma-1)(\mach_{1}^{2}-1)]}{(\gamma+1)^{2}\mach_{1}^{2}}.
\end{eqnarray}
Note that $P_{2} \geq P_{1},\, \rho_{2} \geq \rho_{1}$, and $T_{2}\geq T_{1}$ if $\mach_{1} \geq 1$ (supersonic upstream). In the limit of a very strong shock ($\mach_{1}\rightarrow \infty$),
the density jump is bounded by a finite value $(\gamma+1)/(\gamma-1)$, which is equals 4 if $\gamma=5/3$.
Simultaneously, the flow velocity slows down to 1/4 of $u_{1}$. In this limit, the pressure
and temperature jumps have no bound. 


\bigskip
\subsection{Instabilities}
\subsubsection{Thermal Instability}
\subsubsection{Gravitational Instability: Jean's Instability}
See \S~\ref{subsubsec:Jean}.
\subsubsection{Rayleigh-Taylor Instability}
\subsubsection{Kelvin-Helmholtz Instability}

\bigskip
\subsection{Turbulence}

\bigskip
\subsection{Magneto-hydrodynamics}
%\bibliographystyle{apj}
%\bibliography{citations}
