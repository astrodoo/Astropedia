\chapter{Radiation}

\section{Radiation}

\subsection{Basic Radiation Properties}

\begin{empheq}[innerbox=\fbox]{align}
\textrm{Radiation: Energy transport by electromagnetic fields} \nonumber
\end{empheq}

\subsubsection{Wave-particle duality and the radiative limit}

Since we measure radiation only through the interaction with matter, wave propagation in vacuum is not
all that interesting. When radiation interacts with matter, quantum mechanics can become important.

As Planck found, photon phase space is granular, with a fundamental quantized phase space volume of $h^{3}$.
Photon energy is,
\begin{empheq}[innerbox=\fbox]{align}
E_{\nu} = h\,\nu
\end{empheq}
Photons are relativistic, thus their momentum is directly proportional to their energy:
\begin{empheq}[innerbox=\fbox]{align}
P_{\nu} = \frac{h\nu}{c}
\end{empheq}

\subsubsection{Observables and basic definitions of radiation quantities}

\begin{enumerate}[a)]
   \item Specific intensity or surface brightness: $I_{\nu}$
   
   Define a quantity that describes completely how the measured energy $dE$ depends on $\vec{r},\hat{\vec{k}},\nu,t$ and 
describes the radiation field as completely as possible (neglecting polarization) given the information from the detector:

\begin{equation}\label{eq:intensity}
   I_{\nu}(\vec{r},t,\hat{\vec{k}},\nu) \equiv \frac{dE}{d\nu\,dt\,d\Omega\,dA}
\end{equation}
where the surface $dA$ is taken \textbf{perpendicular} to the direction of the ray $\hat{k}$.

   \item Mean intensity: $J_{\nu}$

   $\Rightarrow~~$ The zeroth moment of $I_{\nu}$ with respect to the polar angle $\cos\theta$:

\begin{equation}
   J_{\nu} \equiv \frac{1}{4\pi} \int_{4\pi} d\Omega\, I_{\nu}
\end{equation}
where the solid angle is,
\begin{equation}
   d\Omega = \sin\theta d\theta d\phi
\end{equation}

   \item Specific flux: $F_{\nu}$

   $\Rightarrow~~$ The first moment of $I_{\nu}$ with respect to $\cos\theta$.

   Energy flux at frequency $\nu$ across a surface $dA$, integrated over all photon directions (coming from both sides of the surface).
\begin{equation}
   F_{\nu} \equiv \int_{4\pi} d\Omega\,\cos\theta\,I_{\nu}
\end{equation}
where $\theta$ is measured relative to the normal of $dA$.

For an unresolved object, we can typically not determine the solid angle spanned by the object. The flux
is therefore the most general quantity we can derive directly from any measurements for the object.

\begin{empheq}[innerbox=\fbox]{align}
F_{\nu}=0~~ \textrm{for isotropic radiation.}
\end{empheq}

Sometimes it is useful to define a photon number flux. Since $e_{\nu}=h\nu$, this is simply
\begin{equation}
   \Phi_{\nu} = F_{\nu}/h\nu
\end{equation}

   \item Total flux: $F$ 
\begin{equation}
   F \equiv \int^{\infty}_{0} d\nu\,F_{\nu} = \left[ \frac{dE}{dt\,dA} \right]  
\end{equation}
   \item Radiation pressure: $P_{rad}$ - Total momentum flux across $dA$
\begin{equation}
   P_{rad} = \int^{\infty}_{0} d\nu \int_{4\pi}d\Omega \cos^{2}{\theta}\frac{I_{\nu}}{c}
\end{equation}
   \item Radiative energy density: $u_{\nu}$ - Amount of radiative energy contained per unit volume
\begin{equation}
   u_{\nu} = \int d u_{\nu} = \int_{4\pi} d\Omega\frac{I_{\nu}}{c} = \frac{dE}{dA\,ds\,d\nu} = \frac{dE}{dV\,d\nu}
\end{equation}
where $ds = cdt$ 

   The energy density of {\bf blackbody radiation} per frequency interval is,
\begin{empheq}[innerbox=\fbox]{align}
    u_{\nu} = \frac{8\pi\nu^{2}}{c^{3}}\frac{h\nu}{e^{h\nu/kT}-1}
\end{empheq}
The specific intensity, eq.~(\ref{eq:intensity}), is related with the energy density via
\begin{equation}
    I_{\nu} = c\frac{d\,u_{\nu}}{d\Omega}
\end{equation}
Since the solid angle of a full sphere is $4\pi$ steradians, the intensity of blackbody radiation is
therefore
\begin{empheq}[innerbox=\fbox]{align}\label{eq:planckf}
    I_{\nu}=\frac{c}{4\pi}u_{\nu}=\frac{2h\,\nu^{3}}{c^{2}}\frac{1}{e^{h\nu/kT}-1} \equiv B_{\nu}
\end{empheq}
which is called as {\bf Planck function}. In terms of wavelength, eq.~(\ref{eq:planckf}) can be re-written by
\begin{equation}
    B_{\lambda} = B_{\nu} \left|\frac{d\nu}{d\lambda}\right| = B_{\nu}\frac{c}{\lambda^{2}}=\frac{2h\,c^{2}}{\lambda^{5}}\frac{1}{e^{hc/\lambda kT}-1}
\end{equation}

For isotropic radiation, $u=3\,P_{rad}$.

\begin{eqnarray}
   P_{rad} &=& \int_{4\pi} d\Omega \cos^2{\theta}\frac{I}{c} = \frac{I}{c}\int^{\pi}_{0} \sin{\theta}\cos^{2}{\theta} d\theta \int^{2\pi}_{0}d\phi \nonumnext
           &=& \frac{2\pi I}{c}\int^{1}_{-1} x^{2} dx = \frac{I}{c} \frac{4\pi}{3}
\end{eqnarray}
where $x \equiv \cos{\theta}$

\begin{equation}
   u = \int_{4\pi} d\Omega \frac{I}{c} = \frac{I}{c}4\pi
\end{equation}

Therefore,
\begin{equation}
   \therefore u=3\,P_{rad} ~~~~\rm for~isotropic~radiation
\end{equation}

   \item Photon density: $n_{\nu}$
number of photons per unit frequency interval per unit volume. 
\begin{equation} 
   n_{\nu} = \frac{u_{\nu}}{h\nu}
\end{equation}
   \item Luminosity: $L_{\nu}$
Specific (spectral) luminosity
\begin{equation}
  L_{\nu} = \oint dA F_{\nu} = \left[ \frac{dE}{dt\,d\nu}\right]
\end{equation}
and total, ``bolometric'' luminosity:
\begin{equation}
  L = \oint dA F = \left[ \frac{dE}{dt}\right]
\end{equation}
   \item Spectral index:$\alpha$

It is often convenient to plot spectra on a log-log plot in frequency, $\log F_{\nu}$ vs. $\log \nu$.
Many emission process produce power-law type spectra over some frequency range,
\begin{equation}
   F_{\nu} \propto \nu^{-\alpha}
\end{equation}
which appear as straight lines in a log-log plot. It is then useful to define a local spectral index
\begin{equation}
   \alpha \equiv -\frac{\partial \log{F_{\nu}}}{\partial \log{\nu}}
\end{equation}
sometimes it is also useful to define a photon index $\Gamma$
\begin{equation}
   \Gamma \equiv -\frac{\partial \log{\Phi_{\nu}}}{\partial \log{\nu}} = -\frac{\partial \left( \log{F_{\nu}} - \log{h\nu} \right)}{\partial \log{\nu}} = \alpha +1 
\end{equation}

   \item $\nu - \lambda$ conversions:
 
Given $\lambda\,\nu = c$, we can express specific quantity $f_{\nu} = df/d\nu$ with respect to wavelength instead:
\begin{equation}
   f_{\nu} = \frac{df}{d\nu} = \frac{df}{d\lambda}\frac{d\lambda}{d\nu} \equiv f_{\lambda}\frac{\lambda^2}{c}
\end{equation}

\end{enumerate}

\bigskip
\subsection{Basic laws}
{\bf Stefan-Boltzmann law:}
\begin{empheq}[innerbox=\fbox]{align}\label{eq:planck}
    u = a\,T^{4}
\end{empheq}
which relates the total energy density of flux of a blackbody to its temperature.
\medskip

\noi {\bf Wien's law:}
The wavelength or frequency of the peak of a blackbody spectrum can be found by taking its derivative and equating to zero
in eq.~(\ref{eq:planck}).
\begin{empheq}[left=\empheqlbrace]{align}
    \lambda_{max}\, T  &=  0.29 ~ {\rm cm\,K}  \\
   h\,\nu_{max} &= 2.8~k\, T
\end{empheq}

\noi {\bf Rayleigh-Jeans approximation:}
At frequencies $\nu$ much lower than the peak (i.e., at photon energies $h\nu \ll kT$) in blackbody spectrum eq.~(\ref{eq:planck}),
\begin{equation}
    B_{\nu}\approx \frac{2\nu^{2}}{c^{2}}\,k\,T ~~~{\rm or,}~~ B_{\lambda}\approx 2\,c\,k\,T\lambda^{-4}
\end{equation}

\noi {\bf Wien tail:}
At frequencies $\nu$ much higher than the peak (i.e., at photon energies $h\nu \gg kT$) in blackbody spectrum eq.~(\ref{eq:planck}),
\begin{equation}
    B_{\nu}\sim e^{-h\nu/kT} ~~~{\rm or,}~~ B_{\lambda} \sim e^{-hc/\lambda kT}
\end{equation}

\noi {\bf Inverse square law:}
\begin{equation}
   F_{\nu} = \frac{L_{\nu}}{4\pi r^{2}}
\end{equation}

\noi {\bf Magnitudes:}

Astronomers have long measured optical fluxes in logarithmic units (magnitudes). This was particularly convenient before
the age of calculators but has stuck around somehow like many archaic scientific habits.

\begin{eqnarray}
   m_{\lambda} &=& -2.5\log{F_{\lambda}} - C_{\lambda} \\
   m_{\nu} &=& -2.5\log{F_{\nu}} - C_{\nu}
\end{eqnarray}
where the constants $C_{\nu,\lambda}$ depend on the specific filter used and the assumed spectrum of the object.

\bigskip
\subsection{Kinds of Radiation}
\subsubsection{hydrogen lines}
The $n$th energy level of the hydrogen atom ($n$=1 is the ground state) is given by the Bohr formula,
\begin{equation}
    E_{n}=-\frac{e^{4}\,m_{e}}{2 \hbar^{2}}\frac{1}{n^{2}} = -13.6\,{\rm eV}\frac{1}{n^{2}}
\end{equation}
The energy difference between two levels is 
\begin{equation}
    E_{n1,n2} = 13.6\,{\rm eV}\left( \frac{1}{n_{1}^{2}} - \frac{1}{n_{2}^{2}} \right).
\end{equation}
The wavelength of a photon emitted or absorbed in a radiative transition between two levels will be
\begin{equation}
    \lambda_{n1,n2} = \frac{h\,c}{E_{n1,n2}}=\frac{911.5 {\rm \angstrom}}{1/n_{1}^{2}-1/n_{2}^2}.
\end{equation}

\begin{table}[ht] 
    \caption{Lyman series \{transition to the ground level (n=1)\}}
  \centering  
  \begin{tabular}{c c c} 
  \hline \hline  
%head  
    label & transition & wavelength \\ 
  \hline 
%data  
    Ly$\alpha$ & 2 $\leftrightarrow$ 1 & 1216 $\angstrom$ \\
    Ly$\beta$  & 3 $\leftrightarrow$ 1 & 1025 $\angstrom$ \\
    Ly$\gamma$ & 4 $\leftrightarrow$ 1 &  972 $\angstrom$ \\
  \hline 
  \end{tabular} 
\end{table}
Up until the {\bf Lyman continuum} (transition from infinity to the ground level ($n=\infty\leftrightarrow1)$,
Ly$_{\rm con}$ = 911.5 $\angstrom$.

\begin{table}[ht] 
    \caption{Balmer series \{transition to the ground level (n=2)\}}
  \centering  
  \begin{tabular}{c c c} 
  \hline \hline  
%head  
    label & transition & wavelength \\ 
  \hline 
%data  
    H$\alpha$ & 3 $\leftrightarrow$ 2 & 6563 $\angstrom$ \\
    H$\beta$  & 4 $\leftrightarrow$ 2 & 4861 $\angstrom$ \\
    H$\gamma$ & 5 $\leftrightarrow$ 2 & 4340 $\angstrom$ \\
  \hline 
  \end{tabular} 
\end{table}
Up until the {\bf Balmer continuum} (transition from infinity to the second level ($n=\infty\leftrightarrow2)$,
Ba$_{\rm con}$ = 3646 $\angstrom$.


\subsubsection{Thermal Bremstrahlung}
to be described
\subsubsection{Synchrotron}
to be described
\subsubsection{21 hydrogen line}
to be described

%\bibliographystyle{apj}
%\bibliography{citations}
