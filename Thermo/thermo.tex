\chapter{Thermo Dynamics}
\section{Thermo Dynamics}

Noting that the volume per unit mass is just the reciprocal of the density, \ie $\rm V = \rho^{-1}$, 
we recognise that the thermal energy equation (\ref{eq:thermalE}) as a statement of {\bf the first law of
thermodynamics:}
\begin{equation}\label{eq:thermo1}
  dU = -P dV + \delta Q,
\end{equation}
that is, the change in the internal energy is equal to the work-pdV done (on the fluid) plus the heat 
added. Note that p,V,U are properties of the fluid (in fact they are thermodynamic state variables) and we 
denote changes in them with the symbol d. In contrast, there is no such property as the heat content and 
so we cannot speak of the change of heat content. Instead, we can only speak of the heat added, and we
therefore use a different notation, \ie $\delta$Q. {\bf The second law of thermodynamics} states that
\begin{equation}\label{eq:eq:thermo2}
  \delta Q = T dS,
\end{equation}
where S is a thermodynamic state variable, the {\it specific entropy} (\ie the entropy per unit mass).
Combining this with the first law, eq. \ref{eq:thermo1}, yields
\begin{equation}\label{eq:thermo}
  dU = TdS - PdV.
\end{equation}

\bigskip
\subsection{Thermal Equilibrium}\label{sec:thermaleq}
T.E.: Medium characterized by a single temperature and every process occurs at the same rate
as its inverse process ($\rm T_{e} = T_{k} = T_{i} = T_{r}$).

\begin{itemize}
   \item $T_{e}$(excite T): Level population following by Boltzmann's distribution. \\
   \begin{equation}
      \frac{n_{2}}{n_{1}} = \frac{g_{2}}{g_{1}} e^{-E/kT_{e}}
   \end{equation}
   \item $T_{k}$(kinetic T): Particle velocity following by Boltzmann-Maxwellian velocity distribution. \\
   \begin{equation}
      f = 4\pi\left( \frac{m}{2\pi k T_{k}} \right)^{3/2} v^{2} e^{-mv^{2}/2 k T_{k}} ~~~~\textrm{or,} ~~~~\frac{1}{2} m v^{2} =\frac{3}{2} k T_{k}
   \end{equation}
   $~~~~~~~~~~~~~~~~~~$ Doppler-Broadening: FWHM = $2 \sqrt{2 ln 2}\, \sigma$, where $\sigma = \sqrt{kT/m}$
   \item $T_{i}$(ionization T): Ionizational fraction by Saha equation
   \begin{equation}
      \frac{n_{z+1}n_{e}}{n_{z}} = \frac{2g_{z+1}}{g_{z}}\left( \frac{2\pi m_{e}k T_{i}}{h^{2}} \right)^{3/2} e^{-\chi/k T_{i}}
   \end{equation}
   \item $T_{r}$(radiational T): Radiation field by Planck function
   \begin{equation}
      B_{\nu} = \frac{2 h \nu^{3}}{c^{2}} \frac{1}{e^{h \nu/kT_{r}} -1}
   \end{equation}
\end{itemize}

L.T.E: ($T_{e} = T_{k} = T_{i} \ne T_{r}$)

cf) Brightness Temperature ($T_{b}$): Radio regime by Rayleigh-Jean's limit
\begin{equation}
   B_{\nu} = \frac{2 h \nu^{3}}{c^{2}} \frac{kT}{h\nu} ~~~~\textrm{or,} ~~~~ T_{b} = B_{\nu} \frac{c^{2}}{2\nu^{2}k}.
\end{equation}

%\bibliographystyle{apj}
%\bibliography{citations}
